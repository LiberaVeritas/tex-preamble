%\usepackage[utf8]{inputenc}    % not with lualatex
%\usepackage[english]{babel}    % not with lualatex

%\usepackage[hybrid, underscores=false, codeSpans=false]{markdown}

%\usepackage{amssymb}    % lualatex



%\usepackage[italicdiff]{physics}

%\usepackage[none]{hyphenat}             % disables hyphenated words at end of paragraph
%\setlength{\parindent}{0pt}             % removes paragraph indent
%\setlength{\parskip}{11pt}              % adds space between paragraphs

%\usepackage{geometry}
%\newlength{\alphabet}
%\settowidth{\alphabet}{abcdefghijklmnopqrstuvwxyz}
%\geometry{textwidth=2.5\alphabet}


%\setlist[enumerate,1]{leftmargin=0pt}   % pushes enum items to the margin

%\usepackage{libertine}
%\usepackage[libertine, vvarbb]{newtxmath}

%\usepackage[space]{erewhon}             % font
%\usepackage[erewhon, vvarbb]{newtxmath} % math font

%\usepackage[scale=0.90]{tgheros}        % sans serif
%\usepackage[scale=0.90,lf]{FiraMono}    % monospace

%\usepackage[T1]{fontenc}   % not with lualatex

\usepackage{microtype}

\usepackage{polyglossia}
\setdefaultlanguage{canadian}

\usepackage{amsmath, amsthm}
\usepackage{mathtools}  % must load before unicode-math and fontspec

\usepackage{enumitem}

\usepackage{lualatex-math}

\usepackage{fontspec}

\usepackage[warnings-off={mathtools-colon, mathtools-overbracket}, math-style=TeX, bold-style=TeX]{unicode-math}

% doc: https://ctan.mirror.globo.tech/macros/unicodetex/latex/unicode-math/unicode-math.pdf
% commands: https://ctan.mirror.rafal.ca/macros/unicodetex/latex/unicode-math/unimath-symbols.pdf
% options: https://texdoc.org/serve/unicode-math/0


\usepackage[CharacterVariant={3,6}, Style={subsetneq}]{fourier-otf}    %varepsilon and varphi

% info: https://ctan.math.washington.edu/tex-archive/fonts/erewhon-math/Erewhon-Math.pdf



\AtBeginDocument{
    \renewcommand{\subset}{\subseteq}
    
    \newcommand{\C}{\symbf C}
    \newcommand{\F}{\symbf F}
    \newcommand{\N}{\symbf N}
    %\newcommand{\PP}{\textbf P}
    \newcommand{\Q}{\symbf Q}
    \newcommand{\R}{\symbf R}
    \newcommand{\Z}{\symbf Z}
    
    \newcommand{\inv}{^{-1}}
    \newcommand{\pre}{^{\preimage}}
    \newcommand{\img}{^{\image}}
    
    %\newcommand{\eps}{\varepsilon}
    
    
    \DeclareMathOperator{\noremptyset}{\text{\O}}
    \renewcommand{\emptyset}{\noremptyset}
    
    \DeclareMathOperator{\Aut}{Aut}
    \DeclareMathOperator{\Char}{char}
    \DeclareMathOperator{\End}{End}
    \DeclareMathOperator{\Gal}{Gal}
    \DeclareMathOperator{\GL}{GL}
    \DeclareMathOperator{\Hom}{Hom}
    \DeclareMathOperator{\Ind}{Ind}
    \DeclareMathOperator{\SL}{SL}
    \DeclareMathOperator{\Spec}{Spec}
    
    \DeclareMathOperator{\coker}{coker}
    \DeclareMathOperator{\id}{id}
    \DeclareMathOperator{\im}{im}
    \DeclareMathOperator{\preimage}{pre}
    \DeclareMathOperator{\image}{img}
    
    \DeclarePairedDelimiter{\paren}{\lparen}{\rparen}
    \DeclarePairedDelimiter{\set}{\lbrace}{\rbrace}
    
    % switch starred version of \set
    \makeatletter
    \let\oldset\set
    \def\set{\@ifstar{\oldset}{\oldset*}}
    \makeatother
    
    % switch starred version of \paren
    \makeatletter
    \let\oldparen\paren
    \def\paren{\@ifstar{\oldparen}{\oldparen*}}
    \makeatother
    
    \newtheorem*{theorem}{Theorem}
    \newtheorem*{lemma}{Lemma}
    \newtheorem*{coro}{Corollary}
    
    \theoremstyle{definition}
    \newtheorem*{definition}{Definition}
    
    \newtheorem*{example}{Example}
    
    
    %\newtheorem{thm}{Theorem}
    %\newtheorem{lemma}[thm]{Lemma}
    %\newtheorem{cor}[thm]{Corollary}
    %\newtheorem{prop}[thm]{Proposition}
    %\newtheorem{claim}[thm]{Claim}
    
    %\theoremstyle{definition}
    %\newtheorem{defn}[thm]{Definition}
    
    %\theoremstyle{remark}
    %\newtheorem{remark}[thm]{Remark}
    %\newtheorem{example}{Example}
}